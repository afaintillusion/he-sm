%for a more compact document, add the option openany to avoid
%starting all chapters on odd numbered pages
\documentclass[12pt]{cmuthesis}

% some useful packages
\usepackage{times}
\usepackage{fullpage}
\usepackage{graphicx}
\usepackage{amsmath}
\usepackage[numbers,sort]{natbib}
\usepackage[backref,pageanchor=true,plainpages=false, pdfpagelabels, bookmarks,bookmarksnumbered,
%pdfborder=0 0 0,  %removes outlines around hyper links in online display
]{hyperref}
\usepackage{subfigure}

% Approximately 1" margins, more space on binding side
%\usepackage[letterpaper,twoside,vscale=.8,hscale=.75,nomarginpar]{geometry}
%for general printing (not binding)
\usepackage[letterpaper,twoside,vscale=.8,hscale=.75,nomarginpar,hmarginratio=1:1]{geometry}

% Provides a draft mark at the top of the document. 
\draftstamp{\today}{DRAFT}

\begin {document} 
\frontmatter

%initialize page style, so contents come out right (see bot) -mjz
\pagestyle{empty}

\title{ %% {\it \huge Thesis Proposal}\\
{\bf An evaluation of the interstitial beat across multisensory modalities for characterization of a meaningful haptic enviro-sensing metronome}}
\author{Nick Pourazima}
\date{May 2018}
\Year{2018}
\trnumber{}

\committee{
Professor Thomas Sullivan \\
Professor Stephen Neely \\
Professor Jesse Stiles \\
}

\support{}
\disclaimer{}

% copyright notice generated automatically from Year and author.
% permission added if \permission{} given.

\keywords{sensorimotor synchronization, vibrotactile perception, haptic, metronome, modality, entrainment}

\maketitle

\begin{dedication}
For
\end{dedication}

\pagestyle{plain} % for toc, was empty

%% Obviously, it's probably a good idea to break the various sections of your thesis
%% into different files and input them into this file...

\begin{abstract}
The interstice is an intervening space. When applied to a rhythmic context, the interstitial beat can be represented by two distinct states; whether energy exists within this small moment in time or if it does not. 

Does filling the space provide an added awareness or preparation for the upcoming onset? Can the gestural motion of the conductor be justified scientifically?

The underlying question when applied to either the daily practice of a trained musician or the innate entrainment, external rhythmic synchronization, of the average human being, is whether the space between the beat matters.

The objective of this work is to display whether a continuous wave, one which leads up to the maximum amplitude of the beat and trails off into a smooth decay, exhibits differentiation from it's instantaneous counterpart in communicating regular or irregular pulses. To quantify this differentiation, an expansive set of analog and discrete tap synchronization test cases spanning the modalities of sight, sound, and touch will be conducted across groups of musicians, amateurs, and non-musicians.

Auxiliary to this work, a haptic wearable design is prototyped and evaluated for optimization of physical spacing with an overarching goal of communicating dynamic changes more effectively.

The work hypothesizes that although rhythmic accuracy is proven to be most effective through discrete audible means [source] there will be improvement shown when the interstitial beat is occupied with a continuous wave across the modality of touch at slower tempo, when space between successive beats is significantly spread apart, as well as throughout the occurrence of unpredictable or dynamically changing events. 

Furthermore, the wearable haptic will provide an inconspicuous and silent gestural system key towards future entrainment studies in expressive performance.

\end{abstract}

\begin{acknowledgments}
First and foremost to my advisor and the original inceptor of this work, Professor Stephen Neely. Our weekly discussions kept me on the right path. Thank you for the guidance and experience you brought to this project. I hope this proves to be exemplary to your design research as well as the framework for future work to come.

Tom,

Riccardo, support

To my roommates and close friends, Mike and Craig. For those long nights of brainstorming possibilities and troubleshooting. Thank you for not only being my think tank but for keeping me inspired and grounded.

Last but not least, a special thank you for the undying love and support of Rachel, for keeping the light at the end of the tunnel shining and maintaining my focus toward the end goal.

\end{acknowledgments}

\tableofcontents
\listoffigures
\listoftables

\mainmatter

%% Double space document for easy review:
%\renewcommand{\baselinestretch}{1.66}\normalsize

% The other requirements Catherine has:
%
%  - avoid large margins.  She wants the thesis to use fewer pages, 
%    especially if it requires colour printing.
%
%  - The thesis should be formatted for double-sided printing.  This
%    means that all chapters, acknowledgements, table of contents, etc.
%    should start on odd numbered (right facing) pages.
%
%  - You need to use the department standard tech report title page.  I
%    have tried to ensure that the title page here conforms to this
%    standard.
%
%  - Use a nice serif font, such as Times Roman.  Sans serif looks bad.
%
% Other than that, just make it look good...


\chapter{Introduction}
\section{Motivation}
I was motivated to write a Phd thesis because I did not want to work directly after finishing my study
\section{Organization}
This thesis is organized as follows, ...\newpage\cleardoublepage
\chapter{Previous Work}
\section{Motivation}
I was motivated to write a Phd thesis because I did not want to work directly after finishing my study
\section{Organization}
This thesis is organized as follows, ...\newpage\cleardoublepage
\chapter{Haptic Design}
\section{Motivation}
I was motivated to write a Phd thesis because I did not want to work directly after finishing my study
\section{Organization}
This thesis is organized as follows, ...\newpage\cleardoublepage
\chapter{Method}
\section{Motivation}
I was motivated to write a Phd thesis because I did not want to work directly after finishing my study
\section{Organization}
This thesis is organized as follows, ...\newpage\cleardoublepage
\chapter{Data Analysis}
\section{Motivation}
I was motivated to write a Phd thesis because I did not want to work directly after finishing my study
\section{Organization}
This thesis is organized as follows, ...\newpage\cleardoublepage
\chapter{Conclusion}
\section{Motivation}
I was motivated to write a Phd thesis because I did not want to work directly after finishing my study
\section{Organization}
This thesis is organized as follows, ...\newpage\cleardoublepage


%\appendix
%\include{appendix}

\backmatter

%\renewcommand{\baselinestretch}{1.0}\normalsize

% By default \bibsection is \chapter*, but we really want this to show
% up in the table of contents and pdf bookmarks.
\renewcommand{\bibsection}{\chapter{\bibname}}
%\renewcommand{\bibpreamble}{This text goes between the ``Bibliography''
%  header and the actual list of references}
\bibliographystyle{plainnat}
\bibliography{register} %your bib file

\end{document}
