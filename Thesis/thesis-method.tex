\chapter{Method}

The main method of data collection for SMS tap based tests involve collection of the time delta between the tap and event onset, called the \textit{asynchrony}. The mean of the asynchonies is typically negative (\textit{NMA}), indicative of the participants anticipation of the beat rather than reaction. Positive asynchonies within the shortest reaction time window (150 ms) are arguably an anticipation of the preceeding stimuli. Furthermore, the standard deviation of the asynchrony $(SD_{asy})$ is an index of stability.  \cite{repp2013sensorimotor}

Other important metrics include the variability and mean of the inter tap interval (\textit{ITI}) and the inter onset interval (\textit{IOI}), or the time between successive beats - measured in milliseconds. Mismatch between the ITI and IOI implies poor synchronization skill from the participant. 

It has been shown previously that professional musicians exhibit a lower ITI variability with percussionists as well as pianists. Surprisingly, both amateurs and non-musicians had no $SD_{asy}$ difference.

A new definition will be put forth for the purposes of this research to quantify the ramp up and decay time of the haptic signal. 

Isochronous vs. nonisochronous ~\cite{polak2016both}

The method which follows is a nature expansion of SMS research to discover whether the haptic modality resonates as much as the audible.

\section{Test Cases}
Test 1:
    Setup:
        All test groups
        Haptic on sleeve, 4 in spacing
        Arduino with FSR to accept finger tap on dominant hand
    Method:
        Modality only:
            Isochronous synchronization:
                1. 45 bpm over 1 minute
                2. 90 bpm over 1 minute
                3. 135 bpm over 1 minute
                4. 180 bpm over 1 minute
            Dynamic synchronization:
                1. variable tempo from 0-45 bpm for 1 minute
                2. variable tempo from 45-90 bpm for 1 minute
                2. variable tempo from 90-135 bpm for 1 minute
                3. variable tempo from 135-180 bpm for 1 minute
        Response to MIDI stimulus:
            Isochronous synchronization:
                3 tracks
            Dynamic synchronization:
                3 tracks
        Response to (beat tracking) audio stimulus:
            Isochronous synchronization:
                3 tracks
            Dynamic synchronization:
                3 tracks

Test 2:
    Setup:
        All test groups    
        2 vibros per arm
        Arduino with FSR to accept finger tap on dominant hand

Test 3:
    Setup:
        For amateurs and prof musicians only
        vibros placed at user discretion   
    Method:
        As a practice tool with primary instrument:
            Isochronous synchronization:
            Dynamic synchronization: 


\section{Tap Onset Latency Evaluation}

A sensorimotor synchronization experiment was conducted to discover how auditory feedback to a tap onset could be presented with minimal latency and responses recorded with the most accuracy. It was found that not only was the auditory response latency the least for the Arduino system using a force sensitive resistor (mean = 0.6 ms, sd = 0.3), but it had missed the fewest taps and recorded the least superfluous responses as compared to a percussion pad with the FTAP and Max MSP systems [Tap Arduino, 1].


\section{Beat Tracking}

Max Patch based on \verb!THIS RESEARCH! 

does this

for the purposing of testing this

\section{Expectations}

If it can be proven for nonmusicians that NMA does not exhibit a linear increase as the IOI increases with the haptic...than, x

According to prior research, expect musicians NMAs to be small and nearly constant as IOI is increased.\cite{repp2013sensorimotor,4}



