% -*- mode:LaTex; mode:visual-line; mode:flyspell; fill-column:75-*-

\chapter{Introduction} \label{secIntro}

Introduction.

\section{Installation instructions}

This template was tested with TeX Live 2017, which includes all required packages~\cite{TUG2017}. Mac users: this is included as part of OSX and TeXShop. After successfully installing TeX Live, compile the PDF file using your favorite build tool (we tested with \verb!make! on OSX).

\section{How to use this template}
Write each chapter as a separate \LaTeX\ file and include them in \verb!thesis-main.tex!. Edit the abstract, acknowledgments, background, title, dedication, and funding files as necessary. Include additional packages in \verb!thesis-packages.tex! and define helpful macros in \verb!thesis-macros.tex!.

\subsection{Algorithms}
Define each algorithm as a separate \LaTeX\ file in the algorithms folder using either the \verb!algorithmicx! or \verb!algpseudocode! packages. For example, see Algorithm~\ref{algTemplate}.

% -*- mode:LaTex; mode:visual-line -*-

\begin{algorithm}
\begin{algorithmic}[1]
\Procedure{Do it}{$N$}
\State Initialize all the things!

\For{t = 1 to N}
\State Do it!
\EndFor
\State \Return $N$
\EndProcedure
\end{algorithmic}
\caption[short caption]{Longer caption}
\label{algTemplate}
\end{algorithm}




\section{Motivation}

Eurythmics, Prof Neely

Is there missing information from the daily practice of a trained musician to an audible metronome. 

Consequentially, the following assumptions arise:


From the work of Jacques Dalcroze, 

What knowledge and/or science is missing?

This research will add another dimension to each sensory modality to resolve the inquiry as to whether filling in the space between the beat, the interstitial, has an impact on rhythmic accuracy with the potential to impact future metronome implementations. 


