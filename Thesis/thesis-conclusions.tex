% -*- mode:LaTex; mode:visual-line; mode:flyspell; fill-column:75-*-

\chapter{Conclusions} \label{chapConclusions}


\section{Future Work}

\subsection{Beat Tracking}

Max Patch based on \verb!THIS RESEARCH! 

does this

for the purposing of testing this

\subsection{Extra-musical Applications}

Parkingson's research

Stroke gait rehabilitation research \cite{holland2014gait}

Through their research they discovered the extension of the application towards those with restricted mobility and morphed the project into the haptic bracelet 4 years later.

Stroke survivors usually suffer from  for the purpose of gait rehabilitation. The results were promising. 

Paper discusses the prior research of audio stimulation and how it yields immediate improvement through entrainment but that they are not lasting

Focussed on triggering the tibialis anterior which contrasts the principles of entrainment, which would utilize rhythmic beats in any sensory modality, regardless of placement. Also haptic masking from leg-to-floor-impact

One patient who was a veteran mentioned that it put a marching sense back into his mind and helped remind him of that sensation of even walking.
