\chapter{Previous Work}
\section{Auditory Advantage}
Extensive research into sensorimotor synchronization has proven that there exists a clear advantage of the discretely timed auditory stimulus as opposed to the visual and tactile counterparts. 

Researchers from the human computer interaction group at the University of Tampere, Finland, conducted an experiment in 2008 to confirm that the instantaneous auditory modality dominates rhythmic perception. Tactile follows close suit with the visual modality being the least suitable for accurately perceiving rhythmic information as well as the most mentally demanding. Instead of the common tap based test, users were given two rhythmic sections and asked to determine whether they were identical or not across modalities as well as combinations of each. [Crossmodal rhythm perception, 3]

Tactile was personally preferred during the test over the other modalities but exploration of pulse length was called upon for further insight. 



Research paper examples/discussion


\section{A Continuous Visual Metronome}
Bouncing ball paper discussion.

Furthermore, an 2014 experiment by the Department of Psychology at Sun Yat-Sen University in Guangdong, China, explored tap synchronization to a visual of a bouncing ball and found that it was not less stable than to an auditory metronome. [Synchronization to a bouncing ball with a realistic motion trajectory, 4]


\section{The Tactile Modality}
A 2016 study by the Department of Psychology at Ryerson University considered whether the auditory advantage persisted across the tactile modality. The experiment was a tap test of non musicians put through a series of simple and complex rhythmic sequences with a varied area of haptic stimulation. In conditions involving a large area of stimulation and simple rhythmic sequences, tactile synchronization closely matched auditory. They proved that if made salient enough, “the accuracy of synchronization to a tactile metronome can equal synchronization to an auditory metronome, further challenging the idea of an auditory advantage over all other modalities for synchronization to discretly timed rhythmic stimuli.” However, auditory won out for synchronization of complex rhythmic sequences.

\section{Commercial Introspection}
Peterson tuner BodyBeat Sync (\$140) seeks to revolutionize the traditional metronome through its extensive coverage of all three modalities with a wearable pulsing vibration unit which claims to “allow musicians to easily internalize the beat and develop a note value relationship both audibly and physically.” [6]

Ramp up/down as well as proof via quantification of this rhythmic internalization are missing.

The Soundbrenner (\$99) is a vibration based metronome using an instantaneous pulse and claims that in freeing the ears, it has “brought the rhythm closer to the body, making it more comfortable and natural to feel the beat and swing of the music instead of chasing the click.” [7]

Similarly, lack of ramp up/down as well as numerical proof.
