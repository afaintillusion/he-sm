% -*- mode:LaTex; mode:visual-line; mode:flyspell; fill-column:75-*-

\chapter{Vibrotactiles} \label{chapAppendix}

The following is a breakdown of available vibrotactiles conducted to inform design perspective.
\begin{enumerate}
    \item Linear electromagetic actuators
    \begin{itemize}
        \item solenoid:
        \begin{itemize}
            \item can leverage resonance, large output for small input
            \item force dependent on position within magnetic field
            \item influenced by device orientation relative to gravity
            \item heats up during use
        \end{itemize}
        \item voice coil:
        \begin{itemize}
            \item linear dynamics yields consistent output, relatively easy to model
            \item \textit{C2 tactor}:
            \begin{itemize}
                \item 7.6mm contactor preloaded against the skin
                \item suspension resonates at 250Hz for maximum perceptibility
            \end{itemize}
            \item \textit{Haptuator}:
            \begin{itemize}
                \item moving magnet design
                \item not meant to touch the skin
                \item optimized to render frequencies above 50Hz
            \end{itemize}
        \end{itemize}
    \end{itemize}
    \item Rotary Electromagnetic Actuators (ERM - eccentric rotating mass)
    \begin{itemize}
        \item simple, reliable, rotate continuously with a constant voltage/current applied
        \item off-center mass affixed to output shaft so that its rotation exerts large radial forces on the body of the motor
        \item couples freq and amplitude of the resulting vibration to the motors rotational speed
        \item small voltage yields weaker vibrations
        \item intrinsic spin-up time could cause delay at the start of the cue
        \item internal static friction can prevent motor from rotating when the applied voltage is very small
    \end{itemize}
    \item Nonelectromagnetic Actuators - Piezoelectric effect
    \begin{itemize}
        \item respond to inputs very quickly and can output arbitrary waveforms
        \item typically require input on the order of 100V
        \item high stiffness of skin creates a need for relatively heavy vibrotactile actuator
        \item most don't have power to move the skin without pushing off a cumbersome mechanical ground
    \end{itemize}
    \item EAP (electroactive polymer) actuators
    \begin{itemize}
        \item uses elastomers rather than ceramics
        \item can achieve larger deformations for lower drive voltages
    \end{itemize}
    \item SMA (shape memory allow) actuators
    \begin{itemize}
        \item remembers original shape
        \item mechanical properties altered in response to temp changes
        \item slow response time, large hystoresis, high energy consumption
    \end{itemize}
    \item Pneumatic systems
    \begin{itemize}
        \item compact, light
        \item require high-pressure air source
        \item struggle to output high-frequency signals
    \end{itemize}
    \item Forced impact
    \begin{itemize}
        \item TacHammer - new technology, specs unknown, hard to acquire
    \end{itemize}
\end{enumerate}
